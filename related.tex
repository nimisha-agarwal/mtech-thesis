\chapter{Related work}
\label{chap:related}

Computer-based educational systems have been in existence for quite a long time. Some of these tools are available online. A notable project is on automated problem generation for education by Microsoft Research \cite{msrprobgen}. The tools in this project are developed for a variety of domains including algebraic proof problems, procedural math problems, sentence completion SAT problems, logic problems, automata construction problems, and board-game problems. 

There are numerous online courses which are also running on MOOC\cite{wiki:mooc} platforms such as Coursera\cite{coursera} and NPTEL\cite{nptel}, which aim to provide quality education to masses across the globe. Another popular instance is the Khan Academy\cite{khanacademy} which was founded by educator Salman Khan in 2006. Apart from providing tutorial videos like the rest of the MOOCs, Khan academy also provides an adaptive web-based exercise system that generates problems for students based on skill and performance\cite{wiki:khanacademy}. Vocabulary.com\cite{vocabulary}, founded by Benjamin Zimmer is an interactive online tool that helps users to improve their vocabulary through repeated exposure to words. Techniques such as flash cards and usages are used to help users remember difficult words. The words are presented in a variety of contexts to help the learning process. More on vocabulary.com can be read on the white paper released by them\cite{vocabwhite}.

Now coming to the domain of compilers, there exists tools which display the result of different phases of compiler in order to teach the phases of Compiler. One of them is Compiler Construction Toolkit\cite{cct}. It has been developed for the lexical analysis and parsing phase of compilers. Users give the desired input to the tool which shows the result obtained from these phases, as output. The tool consists of 2 parts - Compiler Learning Tools and Compiler Design Tools. Compiler Learning Tools involve a NFA to DFA converter, regular expression to finite automata converter and computation of first, follow and predict sets. Compiler Design Tools include scanner generator and parser generator.

The LISA tool \cite{mernik2003educational} founded by Marjan Mernik and Viljem Zumer, helps students learn compiler technology through animations and visualizations. The tool works for 3 phases of compilers - lexical analysis, syntax analysis and semantic analysis. Lexical analysis is taught using animations in DFAs. Similarly for syntax analysis, animations are shown in the construction of syntax trees and for semantic analysis, animations are shown for the node visits of the semantic tree and evaluation of attributes. Students initially observe these animations and understand the semantic functions. Then they slightly modify the semantic functions or find small errors in specifications. In this way, they get to enhance their knowledge of the working of compilers.
