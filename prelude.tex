% prelude.tex
%   - titlepage
%   - dedication (optional)
%   - approval sheet
%   - course certificate
%   - table of contents, list of tables and list of figures
%   - nomenclature
%   - abstract
%============================================================================


\clearpage\pagenumbering{roman}  % This makes the page numbers Roman (i, ii, etc)



% TITLE PAGE
%   - define \title{} \author{} \date{}
\title{A Tool for Teaching Parsing Techniques}
\author{Nimisha Agarwal}
\date{July, 2015}

%  - Roll number, required for title page, approval sheet, and
%    certificate of course work 
\rollnum{13111038} 

%   - The default degree is ``Doctor of Philosophy''
%     (unless the document style msthesis is specified
%      and then the default degree is ``Master of Science'')
%     Degree can be changed using the command \iitbdegree{}
\iitbdegree{Master of Technology}

%   - The default report type is preliminary report.
%      * for a PhD thesis, specify \thesis
\thesis
%      * for a M.Tech./M.Phil./M.Des./M.S. dissertation, specify \dissertation
%\dissertation
%      * for a DIIT/B.Tech./M.Sc.project report, specify \project
%\project
%      * for any other type, use  \reporttype{}
%\reporttype{ReportType}

%   - The default department is ``Unknown Department''
%     The department can be changed using the command \department{}
\department{Computer Science \& Engineering}

%    - Set the guide's name
\setguide{Prof Amey Karkare}
\setguidedept{Department of Computer Science \& Engineering}

%   - once the above are defined, use \maketitle to generate the titlepage
\maketitle

%--------------------------------------------------------------------%
% CERTIFICATE
%     The first page after the title page.
\makecertificate

%--------------------------------------------------------------------%
% COPYRIGHT PAGE
%   - To include a copyright page use \copyrightpage
% \copyrightpage

%--------------------------------------------------------------------%
% ABSTRACT
\begin{abstract}
Interactive Educational Systems is an emerging field in today's world. It compensates the drawbacks of traditional classroom based education by making courses more reachable and manageable. Students are able to absorb knowledge at their own pace and time, rather than synchronize with the speed of instruction in classrooms. 

In this thesis, we have developed an tool for teaching the parsing phase of Compilers. The tool tries to impart knowledge of various parsing techniques to users through generated problems and hints. Problems are generated based on a Context Free Grammar (CFG) given as input. The tool evaluates these generated problems upon receiving the solutions to these problems from the users. Upon evaluation, it generates hint questions when the solutions provided by the users are incorrect. The problems follow a general Multiple Choice Question (MCQ) pattern, where a user is given a problem with a set of possible choices. The hint question generation procedures involve different sorts of algorithms, of which the input string generation algorithm is notable. This algorithm enables creation of an input string based on an incorrect answer provided by the user. Features of this sort help users to learn concepts in parsing in better depth, by understanding the mistakes that they commit.
\end{abstract}

%--------------------------------------------------------------------%
% DEDICATION
%   Dedications, if any.
\begin{dedication}
To my parents
\end{dedication}

% Acknowledgements
\begin{acknowledgments}

I would like to express my gratitude towards my thesis supervisor Prof. Amey Karkare, for providing me his valuable guidance and support throughout this thesis work. It would not have been possible to complete this thesis in due time and come up with an effective tool without his aid. He has provided me with the necessary resources to gather knowledge on building tools for programming languages and compilers. He has also enlightened me on solutions to problems which seemed too difficult to solve for me. I am highly thankful for all the motivation that he has provided to accomplish the thesis work.

I would also like to thank Mohit Bhadade for creating a web-based graphical user interface for the tool developed in this thesis. This has made it possible to demonstrate the real-world capabilities of the tool, by making it usable by students.

\end{acknowledgments}

%--------------------------------------------------------------------%
% CONTENTS, TABLES, FIGURES
\tableofcontents
\listofalgorithms
\addcontentsline{toc}{chapter}{List of Algorithms}

\cleardoublepage
%\phantomsection \label{listoffig}
\listoffigures

\cleardoublepage\pagenumbering{arabic} % Make the page numbers Arabic (1, 2, etc)
