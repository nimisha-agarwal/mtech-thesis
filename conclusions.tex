\chapter{Conclusions}
\label{chap:conclusion}
In this thesis, we have attempted to build a tool to teach parsing techniques in compiler technology to student users, through the use of interactively generated problems. The tool is basically a console based application which can be run off any standard terminal. However, the target audience for which this tool has been designed is not expected to be so familiar with command line interfaces. Hence, a web application had been developed for this tool, in order to make it more usable to students. The tool seamlessly integrated into the web application using the interfaces described in chapter \ref{chap:interface}.

Throughout the construction of the tool, various challenges have been met, especially in developing the algorithm for input string generation in LL parsing. Also, creating an exportable interface which could be used by other applications was also a challenging task. Amidst all these challenges, problem and hint generation techniques have been created for a number of parsing techniques in compiler technology. There are of course deficiencies that the tool still possesses. 

Hence, in future a lot of work can be done to improve this tool, including but not limited to the following:
\begin{enumerate}
\item Generation of input string for the cells of SLR Parsing table, which can be used in hint questions for the incorrectly entered cells of parsing table. 
\item Problem and hint generation for other parsing techniques such as CLR, LALR.
\item Problem and hint generation for other phases of compilers.
\end{enumerate}

Currently, generation of input strings for the cells of SLR Parsing table is in progress. This is similar to the input string generation technique described in chapter \ref{chap:algorithms} section \ref{ssec:inputgen}. Input strings would be generated for incorrect entries filled up by users in the SLR parsing table.